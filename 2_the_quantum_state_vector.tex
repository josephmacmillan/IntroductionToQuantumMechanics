\chapter{The Quantum State Vector}

\section{Basis States}

In the previous chapter I introduced the ket $\ket{\psi}$, and said that it represents a \emph{state} of a quantum mechanical system.  But what exactly is it?  It's hard to define, since by its nature it's abstract -- in some ways, it's meant to represent everything we could possibly know about a particular quantum system.  If we wanted to be technical I could tell you that it's a vector in complex Hilbert space, but I'm not sure that will help. Instead, I think it's best to draw an analogy with a vector space you know well:  vectors in three dimensional position space.  Before we do that, though, I want to also introduce a \emph{different} way of representing a quantum space, called (for reasons we'll see shortly) a ``bra'':
\begin{equation}
\label{eq_bra_definition}
\boxed{
\bra{\psi} \rightarrow \text{ Represents the state of a quantum mechanical system.}
}
\end{equation}
Our job in this chapter is to get at least a good working idea of what exactly a ket and bra actually are and how we use them in quantum mechanics.

I'm sure you recall some of the basics of how we write vectors -- that is, normal old vectors in Cartesian coordinates -- in physics.  We can use the unit vectors
\[
\hat{x}, \quad \hat{y}, \quad \hat{z}
\]
(or, if you prefer, $\hat{i}$, $\hat{j}$, and $\hat{k}$) to specify the three coordinate directions.  These unit vectors can be used to write any other vector in this coordinate system, so for example the vector $\vec{A}$ can be written as
\[
\vec{A} = A_x \hat{x} + A_y \hat{y} + A_z \hat{z},
\]
where $A_x$ and so on are the components of the vector along those directions.  The unit vectors are called \emph{complete} if they can be used to write every possible vector in this way.

In addition, these are \emph{unit} vectors, which means that have a length of exactly one (with no dimensions).  Mathematically we could express this using the dot product,
\[
\hat{x} \cdot \hat{x} = 1, \quad \hat{y} \cdot \hat{y} = 1, \quad \hat{z} \cdot \hat{z} = 1.
\]
We say that a vector is \emph{normalized} if this is the case -- that the dot product with itself is one.

Of course, these unit vectors are also \emph{orthogonal} -- they have 90$^\circ$ between them -- which we could also write in terms of dot products as
\[
\hat{x} \cdot \hat{y} = 0, \quad \hat{y} \cdot \hat{z} = 0, \quad \hat{z} \cdot \hat{x} = 0.
\]
Taken together, we can use the term \emph{orthonormal} for vectors that are both normalized and orthogonal.

The quantum state vectors $\ket{+}$ and $\ket{-}$ play a similar role as the unit vectors, but for spin-1/2 systems; because these states correspond to measuring spin up and spin down along the $z$ axis, these kets form the $S_z$ basis.  And like $\hat{x}$, $\hat{y}$, and $\hat{z}$), they're complete, so that any general spin-1/2 state can be written down in terms of them:
\begin{equation}
\ket{\psi} = a \ket{+} + b\ket{-}.
\end{equation}
The values $a$ and $b$ -- which can be complex numbers, so be careful -- in a loose sense\footnote{Careful with taking this too far; we'll see exactly what $a$ and $b$ tell us soon.} tell you how much of the general state $\ket{\psi}$ is made up of the spin up state ($a$) how much is made up of the spin down state ($b$).

Remember I introduced the bra $\bra{\psi}$ above?  Well, in the $S_z$ basis the state could also be written as
\begin{equation}
\bra{\psi} = a^* \bra{+} + b^* \bra{-}.
\end{equation}
Those stars indicate the \emph{complex conjugate}.  It's important to realize that not only are the two representations of the state -- the ket $\ket{\psi}$ and the bra $\bra{\psi}$ different -- one obviously involves the complex conjugate of the other -- but they're not even the same kind of mathematical object.  That said, either provides a full representation of the same state.

Okay, so what about these funny names, \emph{ket} and \emph{bra}?  They come about because of how we write the \emph{inner product} -- the analog to the dot product we used above for the unit vectors.  We stick the bra first, then the ket, like so:
\[
\braket{\text{bra}}{\text{ket}}.
\]
Can you see what the whole thing spells?  I think this is supposed to be a joke, so feel free to laugh; we're stuck with the notation, though, which Paul Dirac invented in 1939, and which we now call \emph{Dirac notation}. 

With the inner product defined, we can now write out the normalization condition for our basis vectors,
\begin{equation}
\braket{+}{+} = \braket{-}{-} = 1,
\end{equation}
as well as orthogonalization,
\begin{equation}
\braket{+}{-} = \braket{-}{+} = 0.
\end{equation}
Actually, in quantum mechanics, it's not just the basis vectors that are normalized; \emph{every} quantum state must be normalized:
\begin{equation}
\boxed{
\braket{\psi}{\psi} = 1.
}\end{equation}
I know I keep saying this, but we'll see why this must the case later.

\begin{example}[Normalization]  Consider the state 
\begin{equation}
\ket{\psi} = A \left( \ket{+} + 2i \ket{-} \right).
\end{equation}

Is the state normalized?  Check:
\begin{align*}
\braket{\psi}{\psi} &= 1 \\
\Rightarrow A^* \left( \bra{+} - 2i \bra{-} \right) \ A \left( \ket{+} + 2i\ket{-} \right) &=  1
\end{align*}
Expanding out the brackets, using the orthonormality conditions above, gives
\begin{equation}
\label{eq_norm}
5 |A|^2 = 1,
\end{equation}
where $|A|^2 \equiv A^* A$ is the ``complex square''.  Now, the complex square will always be a real number, which we can see easily if we write the the general complex number $A$ in polar form as
\[
A = r e^{i\theta},
\]
where $r$ (the \emph{amplitude}) and $\theta$ (the \emph{phase}) are both real numbers.  Then 
\[
|A|^2 = A^* A = re^{-i\theta} r e^{i\theta} = r^2.
\]
That means the normalization condition above in equation (\ref{eq_norm}) can't give us any information about the phase $\theta$, only the amplitude.  As it will turn out, that's okay -- an overall phase in a quantum state doesn't mean anything physically, and we'll see why in due time -- so by convention we take it to be zero, and set the constant to 
\[
A = \frac{1}{\sqrt{5}}.
\]
With this value of $A$, the state is normalized.
\end{example}

\begin{example}[Inner products]  Consider another state given by
\begin{equation}
\ket{\phi} = \frac{3}{5} \ket{+} - \frac{4}{5} \ket{-}.
\end{equation}
Notice that this state is already normalized (check it!).  What is the inner product between state $\ket{\psi}$ and $\ket{\phi}$?

Using Dirac notation we can write each state in their $S_z$ basis and ``foil'' it out to get
\begin{align*}
\braket{\psi}{\phi} & =  \left( \frac{1}{\sqrt{5}}\bra{+} - \frac{2i}{\sqrt{5}} \bra{-} \right) \left(\frac{3}{5} \ket{+} - \frac{4}{5} \ket{-} \right) \\
& = \frac{3}{5\sqrt{5}} + \frac{8i}{5\sqrt{5}}.
\end{align*}

Notice what happens if we reverse the order of the inner product:
\begin{align*}
\braket{\phi}{\psi} & =   \left(\frac{3}{5} \bra{+} - \frac{4}{5} \bra{-} \right) \left( \frac{1}{\sqrt{5}}\ket{+} + \frac{2i}{\sqrt{5}} \ket{-} \right) \\
& = \frac{3}{5\sqrt{5}} - \frac{8i}{5\sqrt{5}}.
\end{align*}
This is precisely the complex conjugate of the first inner product; in fact, in general for any two state vectors $\ket{\alpha}$ and $\ket{\beta}$ we have
\begin{equation}
\braket{\alpha}{\beta} = \braket{\beta}{\alpha}^*.
\end{equation}
\end{example}

\section*{Problems}
\addcontentsline{toc}{section}{Problems}
\markright{Problems}%

\begin{problem}[Electron spin]
This whole chapter is about electron spin, but of course the electron doesn't really spin around like a top.  But let's pretend for a moment that an electron really is a tiny sphere, of radius
\[
r_e = \frac{e^2}{4\pi \epsilon_0 mc^2}
\]
(the so-called classical electron radius), and that it spins around its axis with angular momentum $S_z = \hbar/2$.  How fast would a point on the ``equator'' be moving?  Does this model make sense? 
\end{problem}

