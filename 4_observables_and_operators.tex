\chapter{Observables and Operators}

\section{Operators and Measurement}

We've already seen lots of examples of \emph{observables} -- the spin components $S_x$, $S_y$, and $S_z$ are all examples of things that we can observe, as are things like position and momentum.  In the language of quantum mechanics: 
\[
\boxed{\text{A physical observable $A$ is represented by an operator $\hat{A}$.}}
\]

Wait, what exactly is an observable?  At the risk of being too simplistic, it's something that \emph{operates} on a quantum state, producing a new state.  Mathematically we can write this operation like
\begin{equation}
\hat{A} \ket{\psi} = \ket{\phi}.\footnote{Although the hat on the operator is standard, many textbooks drop it to reduce clutter in notation.  I think it's important to be able to easily distinguish an observable and an operator at a glance, especially for newcomers.}
\end{equation}
Actually, although in general the operator $\hat{A}$ produces a new state, sometimes the state it operates on is special and \emph{doesn't} change:
\begin{equation}
\label{eq_eigen}
\hat{A} \ket{\psi} = a \ket{\psi}.
\end{equation}
Instead of making a new state, it gives back the same state, but multiplied by a number $a$.  If you're familiar with linear algebra, you might recognize equation (\ref{eq_eigen}) as an \emph{eigenvalue equation}, with $a$ playing the role of the eigenvalue and the ket $\ket{\psi}$ being the eigenvector.   The eigenvalue is very important:
\\

\fbox{
  \parbox{0.9\textwidth}{
    The only possible result of a measurement of an observable $A$ is one of the eigenvalues $a_n$ of the corresponding operator $\hat{A}$.
  }
}

\subsection{The Operator $\hat{S_z}$}

As a good starting point


\section*{Problems}
\addcontentsline{toc}{section}{Problems}
\markright{Problems}%

\begin{problem}[Orthogonality]
Show that the states $\ket{+}_x$ and $\ket{-}_x$ are orthogonal.  Do the same for the states $\ket{+}_y$ and $\ket{-}_y$. 
\end{problem}


