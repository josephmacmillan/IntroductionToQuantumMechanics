\chapter{The Quantum State Vector}

\section{Basis States}

In the previous chapter I introduced the ket $\ket{\psi}$, and said that it represents a \emph{state} of a quantum mechanical system.  But what exactly is it?  It's hard to define, since by its nature it's abstract -- in some ways, it's meant to represent everything we could possibly know about a particular quantum system.  If we wanted to be technical I could tell you that it's a vector in complex Hilbert space, but I'm not sure that will help. Instead, I think it's best to draw an analogy with something you know well:  vectors in three dimensional position space.  Before we do that, though, I want to also introduce a \emph{different} way of representing a quantum space, called (for reasons we'll see shortly) a ``bra'':
\begin{equation}
\label{eq_bra_definition}
\boxed{
\bra{\psi} \rightarrow \text{ Represents the state of a quantum mechanical system.}
}
\end{equation}
Our job in this chapter is to get a good working idea of what a ket and bra actually are and how we use them in quantum mechanics.

I'm sure you recall some of the basics of how we write vectors -- that is, normal old vectors in Cartesian coordinates -- in physics.  We can use the unit vectors
\[
\hat{x}, \quad \hat{y}, \quad \hat{z}
\]
(or, if you prefer, $\hat{i}$, $\hat{j}$, and $\hat{k}$) to specify the three coordinate directions.  These unit vectors can be used to write any other vector in this coordinate system, so for example the vector $\vec{A}$ can be written as
\[
\vec{A} = A_x \hat{x} + A_y \hat{y} + A_z \hat{z},
\]
where $A_x$, $A_y$, and $A_z$ and so on are the components of the vector along those directions.  The unit vectors are called \emph{complete} if they can be used to write every possible vector in this way.

In addition, these are \emph{unit} vectors, which means that have a length of exactly one (with no dimensions).  Mathematically we could express this using the dot product,
\[
\hat{x} \cdot \hat{x} = 1, \quad \hat{y} \cdot \hat{y} = 1, \quad \hat{z} \cdot \hat{z} = 1.
\]
We say that a vector is \emph{normalized} if this is the case -- that the dot product with itself is one.

Of course, these unit vectors are also \emph{orthogonal} -- they have 90$^\circ$ between them -- which we could also write in terms of dot products as
\[
\hat{x} \cdot \hat{y} = 0, \quad \hat{y} \cdot \hat{z} = 0, \quad \hat{z} \cdot \hat{x} = 0.
\]
Taken together, we can use the term \emph{orthonormal} for vectors that are both normalized and orthogonal.

The quantum state vectors $\ket{+}$ and $\ket{-}$ play a similar role as the unit vectors, but for spin-1/2 systems; because these states correspond to measuring spin up and spin down along the $z$ axis, these kets form what we'll call the ''$S_z$ basis.''  And like $\hat{x}$, $\hat{y}$, and $\hat{z}$, they're complete, so that any general spin-1/2 state can be written down in terms of them:
\begin{equation}
\ket{\psi} = a \ket{+} + b\ket{-}.
\end{equation}
The values $a$ and $b$ -- which can be complex numbers, so be careful -- in a loose sense\footnote{Careful with taking this too far; we'll see exactly what $a$ and $b$ tell us soon.} tell you how much of the general state $\ket{\psi}$ is made up of the spin up state ($a$) how much is made up of the spin down state ($b$).

Remember I introduced the bra $\bra{\psi}$ above?  Well, in the $S_z$ basis the state could also be written as
\begin{equation}
\bra{\psi} = a^* \bra{+} + b^* \bra{-}.
\end{equation}
Those stars indicate the \emph{complex conjugate}.  It's important to realize that not only are the two representations of the state -- the ket $\ket{\psi}$ and the bra $\bra{\psi}$ -- different (one obviously involves the complex conjugate of the other), but they're not even the same kind of mathematical object.  That said, either provides a full representation of the same state.

Okay, so what about these funny names, \emph{ket} and \emph{bra}?  They come about because of how we write the \emph{inner product} -- the analog to the dot product we used above for the unit vectors.  We stick the bra first, then the ket, like so:
\[
\braket{\text{bra}}{\text{ket}}.
\]
Can you see what the whole thing spells?  I think this is supposed to be a joke, so feel free to laugh; we're stuck with the notation, though, which Paul Dirac invented in 1939, and which we now call \emph{Dirac notation}. 

With the inner product defined, we can now write out the normalization condition for our basis vectors,
\begin{equation}
\braket{+}{+} = \braket{-}{-} = 1,
\end{equation}
as well as orthogonalization,
\begin{equation}
\braket{+}{-} = \braket{-}{+} = 0.
\end{equation}
Actually, in quantum mechanics, it's not just the basis vectors that are normalized; \emph{every} quantum state must be normalized:
\begin{equation}
\boxed{
\braket{\psi}{\psi} = 1.
}\end{equation}
I know I keep saying this, but we'll see why this must the case later.

\begin{example}[Normalization]  Consider the state 
\begin{equation}
\ket{\psi} = A \left( \ket{+} + 2i \ket{-} \right).
\end{equation}

Is the state normalized?  Check:
\begin{align*}
\braket{\psi}{\psi} &= 1 \\
\Rightarrow A^* \left( \bra{+} - 2i \bra{-} \right) \ A \left( \ket{+} + 2i\ket{-} \right) &=  1
\end{align*}
Expanding out the brackets and using the orthonormality conditions above gives
\begin{equation}
\label{eq_norm}
5 |A|^2 = 1,
\end{equation}
where $|A|^2 \equiv A^* A$ is the ``complex square.''  Now, the complex square will always be a real number, which we can see easily if we write the the general complex number $A$ in polar form as
\[
A = r e^{i\theta},
\]
where $r$ (the \emph{amplitude}) and $\theta$ (the \emph{phase}) are both real numbers.  Then 
\[
|A|^2 = A^* A = re^{-i\theta} r e^{i\theta} = r^2.
\]
That means the normalization condition above in equation (\ref{eq_norm}) can't give us any information about the phase $\theta$, only the amplitude.  As it will turn out, that's okay -- an overall phase in a quantum state doesn't mean anything physically, and we'll see why in due time -- so by convention we take it to be zero, and set the constant to 
\[
A = \frac{1}{\sqrt{5}}.
\]
With this value of $A$, the state is normalized.
\end{example}

\begin{example}[Inner products]  Consider another state given by
\begin{equation}
\ket{\phi} = \frac{3}{5} \ket{+} - \frac{4}{5} \ket{-}.
\end{equation}
Notice that this state is already normalized (check it!).  What is the inner product between state $\ket{\psi}$ and $\ket{\phi}$?

Using Dirac notation we can write each state in their $S_z$ basis and ``foil'' it out to get
\begin{align*}
\braket{\psi}{\phi} & =  \left( \frac{1}{\sqrt{5}}\bra{+} - \frac{2i}{\sqrt{5}} \bra{-} \right) \left(\frac{3}{5} \ket{+} - \frac{4}{5} \ket{-} \right) \\
& = \frac{3}{5\sqrt{5}} + \frac{8i}{5\sqrt{5}}.
\end{align*}

Notice what happens if we reverse the order of the inner product:
\begin{align*}
\braket{\phi}{\psi} & =   \left(\frac{3}{5} \bra{+} - \frac{4}{5} \bra{-} \right) \left( \frac{1}{\sqrt{5}}\ket{+} + \frac{2i}{\sqrt{5}} \ket{-} \right) \\
& = \frac{3}{5\sqrt{5}} - \frac{8i}{5\sqrt{5}}.
\end{align*}
This is precisely the complex conjugate of the first inner product; in fact, in general for any two state vectors $\ket{\alpha}$ and $\ket{\beta}$ we have
\begin{equation}
\boxed{
\braket{\alpha}{\beta} = \braket{\beta}{\alpha}^*.
}
\end{equation}
\end{example}

\section{Matrix Notation}

Dirac notation is fun and expressive, but there's another way to write out a quantum state that can be a little more practical for calculations.  Take a look at the form of some general state, which in Dirac notation is
\begin{equation}
\ket{\psi} = a \ket{+} + b \ket{-}.
\end{equation}

Now, the kets $\ket{+}$ and $\ket{-}$ are the basis states as we've mentioned, and $a$ and $b$ are two complex numbers that tell you -- again, loosely -- how much of each basis state makes up the state $\ket{\psi}$.  We can ``pick out'' these numbers with an inner product; for example,
\[
\braket{+}{\psi} = \bra{+} \left( a \ket{+} + b \ket{-} \right) = a \braket{+}{+} + b \braket{+}{-} = a.
\]
Notice how the bra $\bra{+}$ acts past the numbers $a$ and $b$ to stick to the kets, and that to make the last equality we used orthonormality. Likewise, the inner product 
\[
\braket{-}{\psi} = b.
\]

As long as we know the basis we're in, any state can be specified by only those two numbers $a$ and $b$.  So let's use just them and write the state as a \emph{column vector},
\begin{equation}
\ket{\psi} \to \begin{pmatrix} a \\ b \end{pmatrix}.
\end{equation}
I'm using an arrow ($\to$) rather than an equal sign since we're mixing notations here; the ket $\ket{\psi}$ is written in Dirac notation, and the column vector is what we'll call \emph{matrix notation}.  We could also write the column vector as
\begin{equation}
\ket{\psi} \to \begin{pmatrix} \braket{+}{\psi} \\ \braket{-}{\psi} \end{pmatrix},
\end{equation}
which mixes the two notations even further but makes it clear what basis the state is written in.  It also makes it clear that the basis states themselves are just unit column vectors:
\begin{equation}
\ket{+} \to \begin{pmatrix} 1 \\ 0 \end{pmatrix} \quad \text{and} \quad \ket{+} \to \begin{pmatrix} 0 \\ 1 \end{pmatrix}.
\end{equation}

But what about the bra -- how do we represent that?  With a bit of thought, you might agree that they should be \emph{row vectors}, so that the general state is written as
\begin{equation}
\bra{\psi} \to \begin{pmatrix} a^* & b^* \end{pmatrix}
\end{equation}
or 
\begin{equation}
\label{eq_row}
\bra{\psi} \to \begin{pmatrix} \braket{\psi}{+} & \braket{\psi}{-} \end{pmatrix}.
\end{equation}
We're really leveraging all this notation -- the row vector elements are the complex conjugate of the numbers $a$ and $b$, so in equation (\ref{eq_row}) I've swapped the order of the two states in the inner product.

\begin{example}[Inner products, again]
Let's go back to the previous example and recompute the inner product $\braket{\psi}{\phi}$ but using matrix notation.  Each state can now be written as 
\[
\ket{\psi} = \begin{pmatrix} 1/\sqrt{5} \\ 2i/\sqrt{5} \end{pmatrix} \quad \text{and} \quad \ket{\phi} \to \begin{pmatrix} 3/5 \\ -4/5 \end{pmatrix}.
\]
Remember the bra is the row vector (with complex conjugate elements) and the ket the column, so the inner product works out perfectly:
\[
\braket{\psi}{\phi} \to \begin{pmatrix} 1/\sqrt{5} & -2i/\sqrt{5} \end{pmatrix} \begin{pmatrix} 1/\sqrt{5} \\ 2i/\sqrt{5} \end{pmatrix} = \frac{3}{5\sqrt{5}} + \frac{8i}{5\sqrt{5}}.
\]
This is of course the same result we got last time, but if you're like me you find it a little easier to multiply matrices than foil out in Dirac notation.  

\end{example}

\section*{Problems}
\addcontentsline{toc}{section}{Problems}
\markright{Problems}%

\begin{problem}[Spin-1/2 States]
Consider these different spin-1/2 states, given in Dirac notation by
\begin{align*}
\ket{\psi_1} &= 3 \ket{+} + 4\ket{-} \\
\ket{\psi_2} &= \ket{+} + 2i \ket{-} \\
\ket{\psi_3} &= 3 \ket{+} - e^{i\pi/3}\ket{-} \\
\ket{\psi_4} &= 3 \ket{+} - 5i\ket{-}.
\end{align*}
\begin{enumerate}[label=(\alph*)]
\item Normalize each state.
\item Write each state in matrix notation.
\item Write the bra of each ket; write the bras in matrix notation, too.
\item Find a (normalized) ket that is orthogonal to each state.
\item Compute the following inner products, using either Dirac notation or matrix notation:  $\braket{\psi_1}{\psi_2}$, $\braket{\psi_3}{\psi_4}$, $\braket{\psi_4}{\psi_1}$, and $\braket{\psi_3}{\psi_2}$.
\end{enumerate}
\end{problem}


\begin{problem}[A Three-State System]
So far we've only looked at the fairly simple two-state spin-1/2 system.  For this problem, consider a system which has \emph{three} orthonormal basis states, given by
\[
\ket{a_1}, \ket{a_2}, \text{ and }  \ket{a_3}.
\]
\begin{enumerate}[label=(\alph*)]
\item What would these kets look like in matrix notation?
\item Normalize the state
\[
\ket{\psi_1} = \ket{a_1} - 2 \ket{a_2} + 5 \ket{a_3}.
\]
\item Take the inner product of $\ket{\psi_1}$ with the state 
\[
\ket{\psi_2} = i\ket{a_1} + 3 \ket{a_2} - 2\ket{a_3}.
\]
(you'll have to normalize that state first, of course).
\end{enumerate}
\end{problem}

