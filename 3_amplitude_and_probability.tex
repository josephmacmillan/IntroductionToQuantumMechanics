\chapter{Amplitude and Probability}

\section{The $S_x$ and $S_y$ Spin States}

\section{Calculating Probability}

\section{The Stern-Gerlach Experiments Revisited}

\section{Indeterminant States}

\section*{Problems}
\addcontentsline{toc}{section}{Problems}
\markright{Problems}%

\begin{problem}[Spin-1/2 States]
Consider these different spin-1/2 states, given in Dirac notation by
\begin{align*}
\ket{\psi_1} &= 3 \ket{+} + 4\ket{-} \\
\ket{\psi_2} &= \ket{+} + 2i \ket{-} \\
\ket{\psi_3} &= 3 \ket{+} - e^{i\pi/3}\ket{-} \\
\ket{\psi_4} &= 3 \ket{+} - 5i\ket{-}.
\end{align*}
\begin{enumerate}[label=(\alph*)]
\item Normalize each state.
\item Write each state in matrix notation.
\item Write the bra of each ket; write the bras in matrix notation, too.
\item Find a (normalized) ket that is orthogonal to each state.
\item Compute the following inner products, using either Dirac notation or matrix notation:  $\braket{\psi_1}{\psi_2}$, $\braket{\psi_3}{\psi_4}$, $\braket{\psi_4}{\psi_1}$, and $\braket{\psi_3}{\psi_2}$.
\end{enumerate}
\end{problem}


\begin{problem}[A Three-State System]
So far we've only looked at the fairly simple two-state spin-1/2 system.  For this problem, consider a system which has \emph{three} orthonormal basis states, given by
\[
\ket{a_1}, \ket{a_2}, \text{ and }  \ket{a_3}.
\]
\begin{enumerate}[label=(\alph*)]
\item What would these kets look like in matrix notation?
\item Normalize the state
\[
\ket{\psi_1} = \ket{a_1} - 2 \ket{a_2} + 5 \ket{a_3}.
\]
\item Take the inner product of $\ket{\psi_1}$ with the state 
\[
\ket{\psi_2} = i\ket{a_1} + 3 \ket{a_2} - 2\ket{a_3}.
\]
(you'll have to normalize that state first, of course).
\end{enumerate}
\end{problem}

